%	The Common RISC Platform -- Desktop Edition
%
%	CHANGES:
%		27-12-20: jcalligeros - Intial version
%

\documentclass[12pt]{report}
\usepackage[a4paper, portrait, margin=25mm]{geometry}
\usepackage[parfill]{parskip}
\usepackage{yhmath}
\usepackage{listings}
\lstdefinestyle{routines}{
	language=C,
	numbers=left,
	stepnumber=1,
	numbersep=10pt,
	tabsize=4,
	showspaces=false,
	showstringspaces=true
}
\title{The Common RISC Platform\\ \large{Desktop Edition}}

\begin{document}

\maketitle

\newpage

\tableofcontents
\newpage

%----------------------------------------------------------------------------------------------------------------------%
\section{Preface}
\textit{Adapted from the first public post in the CRP development blog}
It cannot have gone unnoticed by many that the PC as a platform is under siege. For the past few years, development on
RISC-V and ARM based SoCs has continued at a blinding pace, while improvements in the x86 world have somewhat stagnated.
It is also easy to see how vendors of such chips have been using this opportunity to either develop or reinforce existing
closed-loop ecosystems. One need only look at Apple’s recent release of its M1-based Macs for an example of this.

Thus, I have come to the conclusion that the PC as we know will be gone by the end of the decade. Rather than simply lie down
and let the likes of Apple push us into a hellish dystopia of closed platforms, planned obsolescence and soldered-on-board
mass storage, I believe that it is in the industry’s best interests if a free, open and fair standard platform were to be developed.

It would surprise many to learn that there once was a time when users could replace their P5 Pentiums with an AMD K6 or Cyrix
6x86 without changing anything else about the system. These days of course are long gone, giving way to the mess of chipset and
socket changes we’re more familiar with today.

While the PC itself may be heading into its sunset years, this does not mean that open computing is dead. Far from it, in fact.
I have noticed an ever-increasing interest from both the general populace and industry players in open-source hardware and ISAs like
Power and RISC-V. Unfortunately, development seems to be quite stagnant in terms of developing products the average consumer would
actually want to use.

Most RISC-V based boards are little more than glorified prototyping boards, with next to no extensibility or consumer-friendly features.
On the other end of the spectrum we have the burgeoning market of Power ISA machines. While some of these are definitely a step in the
right direction, software support for Power ISA is still very limited, and despite being RISC in name, the ISA itself is extremely
complex and not well suited for simple/efficient consumer devices. Not only this, but many of the currently existing systems are focused
more on the server/HPC market, foregoing I/O and design choices that would make them more appealing to the consumer market.

In my opinion, this fragmentation of the industry is just as bad as the closed-loop ARM ecosystems of the mobile space. At the end
of the day, development work is still divided and consumers are still forced into a platform catered to a niche.

In my mind, the only way forward for the client compute industry is for an open, flexible and extensible platform to be adopted
industry-wide. A platform that closely emulates the PC philosophy, but simplifies and modernises the technical underpinnings to prepare
for the future of computing post-PC. A platform that defines a common set of hardware and standards to allow both software and
hardware vendors the freedom to develop products for a standard lowest common denominator, and to reach a broad market. A platform that
gives the user total and utter freedom over their machine.

We are at a crossroads in the industry, and I strongly believe that a user-led revolt against the status quo is what’s needed to
demonstrate that the current trend towards proprietary ecosystems is not acceptable to users, nor will it be beneficial in the
long-term to the vendors’ bottom lines.
\newpage


\section{Standard Assumptions and Rules}
For the entirety of this document, it is assumed that the reader is familiar with the following standards and concepts:

\begin{itemize}
	\item The RISC-V Instruction Set Architecture
	\item ATX 12VO
	\item PCI Express
	\item NVM Express
	\item Coreboot
	\item ATX and derivative motherboard form factors
	\item Coreboot
	\item ACPI power management
\end{itemize}

If the reader is only interested in the Common RISC Platform as an academic exercise, then intricate knowledge of these
standards is not required; a fundamental understanding of their basic design is sufficient. However, this document
assumes that prospective Platform vendors are already familiar with the intricacies of these standards. They will not
be explained or elaborated upon other than to prescribe specific operating conditions required by the Platform.

The only exception to this rule is for concepts that require diagrammatic explanation, such as how the SSI board form
factors relate to other physical connectors that are part of the Platform specification.

This document describes the `base' Common RISC Platform, which we have designed to take the place of the modern consumer
desktop Personal Computer, from the low-power OEM market up to and including the enthusiast gaming market. We are very
interested in developing variants of the Platform for the HEDT and portable device markets. All market stakeholders are
invited to contribute proposals for extensions and modifications to adapt the base specification to be better suited for
these markets.
%----------------------------------------------------------------------------------------------------------------------%
%----------------------------------------------------------------------------------------------------------------------%

\section{Glossary of Terms}
\newpage
%----------------------------------------------------------------------------------------------------------------------%
%----------------------------------------------------------------------------------------------------------------------%
%----------------------------------------------------------------------------------------------------------------------%

\chapter{A Rationale}
\newpage

%----------------------------------------------------------------------------------------------------------------------%
\section{Background}
In 2006, then-CEO of Apple Steve Jobs got on stage at the Worldwide Developers Conference and announced that the company
would be switching over the Macintosh to IA-32 from PowerPC starting from 2007. This was done in response to the enormous
performance gains Intel was making at the time; the company had put the failed NetBurst microarchitecture to rest, and
had just released chips based on Core, which was a derivative of the older P6 microarchitecture. Core-based chips were
significantly more efficient, offered higher performance and were cheaper than the AIM Alliance's PowerPC offerings. IBM
and Motorola were simply unable to continue increasing the performance of their PowerPC chips to match.

This Intel transition was the final nail in the coffin for the RISC desktop. The mid-90s and early 2000s saw fierce
competition between RISC architectures for the desktop market from Acorn's ARM and clones thereof, to SGI and their MIPS
based workstations, to Tadpole and their SPARC-based devices. By the time Apple had announced its transistion, Macs were
the last non-x86 client devices left on the market (ignoring the portable device market, which had by this time been
completely dominated by ARM).

However, beginning around 2013, performance gains started drying up. Desktop chips were no longer gaining 30\% or 40\%
performance relative to their predecessors. The high-end market stagnated, with performance uplifts getting down to single-digit
percentages, all the while prices crept ever upward. This is due to the performance gains of these large CISC chips being
almost entirely the result of process node shrinkages rather than any architectural optimisations. This is evidenced even
as recently as the Intel *lake series of chips, which were all built on Intel's 14nm process and saw performance gains that
almost exactly correlated with the net increase in clock speed, indicating that there was very little architecture-level
optimisation made.

%----------------------------------------------------------------------------------------------------------------------%
\section{ARM's March Forward}
Despite the failure of any RISC-based architecture to penetrate the desktop market, by 2005 ARM chips were being deployed
in around 98\% of mobile devices. Given the simplicity of the ARM instruction set, fully-featured ARM cores could be
implemented with far fewer transistors than a given CISC IA-32 machine. This made it perfect for low-power deployments,
such as the microcontrollers found in the PDAs, digital organisers and feature phones of the era.

In a sort of positive-feedback loop, as ARM became entrenched in the portable market, ARM-based designs were increasingly
optimised for performance-per-Watt, rather than simply raw performance. So highly optimised has the ARM ecosystem become
that not even Intel, with all its financial horsepower, could compete in compute efficiency. However, in recent times the
raw performance lead of x86 chips has been eroded. Apple's A-series mobile SoCs are now able to outperform most of the
x86-based low-power designs from Intel and AMD, due in part to the embracement of heterogeneous compute principles and
the inherent efficiency advantages of the RISC architecture.

Tim Cook has announced that the Macintosh will be transistioning away from IA-32 and to ARM, starting with the MacBooks.
As with the PPC transition, this is due to the performance, efficiency and value stagnation of x86-based offerings. The difference,
however, is that this time Apple is setting the trend, not following it. Apple is the first chip vendor to put forward
an ARM-based SoC that can truly compete with x86 offerings. They will not be the last.

%----------------------------------------------------------------------------------------------------------------------%
\section{Learning From History}
As part of the competition between RISC and CISC based machines that defined the industry in the 90s, many standards were
developed to compete with the PC platform. The Advanced Computing Environment was a consortium of system vendors that developed
the Advanced RISC Computing (ARC) platform to compete with Wintel -- the de-facto PC platform of Intel IA-32 microprocessors
running various incarnations of Microsoft Windows. Many vendors signed on to develop machines for ARC, but no machine was
ever produced that was fully ARC compliant. There are many reasons for ARC's failure. However, as with the fall of the
Western Roman Empire or the outbreak of World War I, there is always a catalysing incident.

The Advanced Computing Environment consortium had settled upon the MIPS architecture as the platform standard. This was
done as MIPS Computer Systems was at the time a vendor-neutral fabless design company. However, in 1992, SGI purchased
MIPS Computer Systems, which caused a major conflict of interest for the other ARC vendors. Could the newly-incorporated
MIPS Technologies be trusted to remain neutral, being a wholly owned subsidiary of SGI? Would SGI receive preferential
licensing fee arrangements or preferential access to MIPS's engineering resources? History tells us that the answer was
yes, however the ACE consortium barely lasted long enough to find out for themselves. Most vendors quickly abandoned ship
and returned to designing Wintel machines.
\rule{16cm}{0.4pt}

\small
\textbf{Aside:} ARC's enduring legacy was in its influence over the design of many low-level
components of Windows NT, which was originally intended for ARC. The NT bootloader and kernel before NT6.0 used the
hardware naming conventions and mapping scheme provided by the ARC system firmware, even on IBM PC clones. The first stage
of the NT BIOS bootloader was nothing more than an ARC firmware emulator implemented in x86 assembly and C. \\
\normalsize
\rule{16cm}{0.4pt}

%----------------------------------------------------------------------------------------------------------------------%
\section{Fair, Open, and Free}
We are of the belief that competition and innovation can only best occur when there is a fair, free and open platform
on which companies can compete. Anything less stifles consumer choice and encourages vendors to develop closed systems
that reinforce lock-in. This only leads to a fragmented market in which consumers have no power to influence vendors.
A locked in consumer is a voiceless consumer. One need only look at the anti-competitive and anti-consumer behaviour
practiced by many major chip vendors. A chip vendor should not try to sell an ecosystem, and a platform vendor should
not try to lock consumers in to using their chips, and their chips alone.

Thus, we present the Common RISC Platform. The CRP aims to be a feature-complete platform based on existing
open standards found in the current PC platform, unified and simplified to provide a platform that is as simple or as
flexible as the task requires. The CRP utilises standards such as PCI Express, NVM Express, and ATX12VO to provide a
simple, yet highly flexible and customisable platform. The CRP mandates certain high and low level commonalities
between all implementations to ensure full compatibility, however implementations are also highly customisable in terms
of feature implementation to ensure vendors can cost-effectively implement competitive CRP systems.

We intend to fully open-source the Common RISC Platform under a strong copyleft license. We do this in the hopes
that the industry recognise the advantages of a universal, vendor-agnostic platform, and that making its adoption as
painless as possible encourages vendors to compete not on the effectiveness of their Bernaysian marketing, but on
the strengths of their engineering teams.

\newpage
%----------------------------------------------------------------------------------------------------------------------%
%----------------------------------------------------------------------------------------------------------------------%
%----------------------------------------------------------------------------------------------------------------------%

\chapter{Main System Package}
\large
In the Common RISC Platform, the Main System Package replaces the traditional CPU. All CRP-compliant MSPs \textbf{must}
implement:
\begin{itemize}
	\item At least one RISC-V logic core
	\item A DDR5 memory controller
	\item A PCI Express Root Complex
	\item A USB4 Host Controller
\end{itemize}
This chapter describes how these components should be implemented.
\normalsize
\newpage

%----------------------------------------------------------------------------------------------------------------------%
\section{Nomenclature}
The Main System Package can be thought of in the same way a modern `CPU' would be. The change in nomenclature reflects the
changing role of the CPU in a modern desktop platform. The modern CPU package typically integrates the various I/O controllers
traditionally found on the motherboard's northbridge, as well as increasing numbers of fixed-function units to accelerate
specific tasks like cryptography. We initially considered prescribing the return of the Northbridge to handle I/O and memory
access, however the vast increase in memory latency and the resultant performance hit would be unacceptable, even when
traded against the decrease in system complexity.

Hence, we believe that `CPU' is a poor monicker for such a device going forward, as modern `CPUs' are typically neither
central to many compute tasks nor do they solely process information. We did not want to use the term `SoC' for similar
reasons -- the `CPUs' of today are already employing advanced packaging techniques that separate functional blocks into
unique ICs that are interconnected. Likewise, the term `SoP' is equally inaccurate as many devices will indeed remain
integrated, monolithic ICs. Additionally, neither term truly reflects devices that more closely resemble traditional CPUs.

\section{Minimum Logic Implementation}
Common RISC Platform MSPs shall contain \textbf{at least one} RISC-V core that fully implements the \textbf{RV64GC} Instruction
Set. This is the superset of RV64I, M, A, F, D, Zicsr, Zifencei, and C instruction sets. This is the minimum instruction
set necessary to cleanly boot a Linux userspace targeted at RISC-V. Core vendors may choose to implement other RISC-V
extensions based on their target market, however subsets are strictly forbidden. All other design decisions are at the
sole discretion of the designer.

Devices that employ microcode as part of their design must store the code on package. As the Platform Initialisation
system is designed to be interoperable, MSP-specific microcode cannot be stored to and loaded from the ROM on the
motherboard as is possible with the current PC platform.

\rule{16cm}{0.4pt}\\
\small
\textbf{IMP. DEP. \#1\:}
Core/MSP designers are free to implement additional RISC-V Instruction Sets as they see fit, however subsets are
strictly forbidden.\\
\textbf{IMP. DEP\#2\:}
Vendors of logic devices that use microcode are encouraged to make that code publicly available for audit. Vendors are
also encouraged to store microcode on EEPROM, and provide utilities that allow this to be flashed with updated code.\\
\normalsize
\rule{16cm}{0.4pt}
\newpage

\section{Memory Controller}
The Common RISC Platform uses JEDEC DDR5 SDRAM for main system memory. In keeping with current desktop platforms, the MSP
shall implement a dual-channel DDR5 memory controller for communicating with the system memory. Implementation details
are at the discretion of the vendor.

\section{System Communication}
The Common RISC Platform prescribes a univeral MSP socket, CRP-1. All compliant MSPs must be electrically and physically
compatible with the CRP-1 socket, as described under \textbf{link to socket chapter}.

The MSP shall accept two voltage inputs -- $V_{DDC}$ and $V_{DDIO}$. MSP-VRM communications are explained in detail
in \textbf{POWER DELIVERY CHAPTER GOES HERE}. The MSP will, at stock settings, draw no more than 85W from the socket.

The MSP must expose a PCI Express Root Complex. Most off-package communication with other system components must be done
through the PCI Express fabric. The Root Complex must expose 40 lanes to the system. 36 of these lanes may be flexibly
assigned by motherboard vendors for any PCIe socket, and the remaining 4 are reserved for interfacing with
the Platform Hub. The CRP-1 pinout allocates specific pins for each category of lane. Other interconnects must communicate
with the MSP through the Platform Hub. This is described in depth in \textbf{MOTHERBOARD CHAPTER GOES HERE}.

The MSP must also expose a USB4 Host Controller. MSPs that implement a GPU on package must use USB4 DisplayPort Alternate
Mode for display output. All boards, regardless of other design decisions, must implement at least one USB Type-C connector
on the rear I/O panel connected directly to the MSP Host Controller such that any board can be fully compatible with
any MSP, including those with integrated graphics.

Out of band signalling for the purposes of platform initialisation, power management and inter-chip communications shall
be accomplished through the eSPI bus, or via I2C.
%----------------------------------------------------------------------------------------------------------------------%
%----------------------------------------------------------------------------------------------------------------------%
%----------------------------------------------------------------------------------------------------------------------%


\chapter{Platform Initialisation}
\newpage

%----------------------------------------------------------------------------------------------------------------------%

\section{Specifications}
All CRP-compliant motherboards shall use \textbf{coreboot} for platform initialisation. Systems must support all coreboot
payloads, and no CRP-compliant device shall prevent the execution of any payload under any circumstances. The coreboot
image shall be stored on a \textbf{socketed} 256Mb EEPROM chip. Additionally, compliant boards shall not prevent the user from
replacing the manufacturer-provided PI firmware with any compatible firmware image from third parties.

Motherboard vendors are encouraged to open source their coreboot images  

\rule{16cm}{0.4pt}\\
\small
\textbf{IMP. DEP. \#3\:}
A CRP-compliant coreboot may or may not provide a pre-payload user interface that allows for platform settings to be
customised, e.g. a coreboot may have facilities that allow the user to overclock their MSP and memory.\\
\normalsize
\rule{16cm}{0.4pt}

\newpage

%----------------------------------------------------------------------------------------------------------------------%
%----------------------------------------------------------------------------------------------------------------------%
%----------------------------------------------------------------------------------------------------------------------%


\chapter{The CRP-1 Socket}
\large
The CRP-1 socket defines a standard interconnect for all MSPs. This chapter describes the physical and electrical
characteristics of the CRP-1 socket.
\normalsize
\newpage
%----------------------------------------------------------------------------------------------------------------------%
\section{Physical Description}
CRP-1 is a Pin Grid Array socket with 2155 pin positions, each with a diameter of 0.5mm and a pitch of 1mm. CRP-1 is arranged
in a square grid of 49 pins by 49 with multiple key positions to ensure that packages can only be inserted correctly. 


\end{document}